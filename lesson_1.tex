\documentclass{tstextbook}

\begin{document}
	
	%---------------------------------------------------------------------------
	% Chapters
	%---------------------------------------------------------------------------
	
	%---------------------------------------------------------------------------
	\chapter{Lesson 1}
	
	\section{Alphabet and transcription}
	
	\textcolor{orange}{\LARGE A B C D E F G H I J K L M N O P Q R S T U V W X Y Z}
	
	\begin{theorem}[Transcription]
		\label{th:transcription}
		\index{trancription}
		Основные правила транскрипции:
		\begin{align*}
			CH &\Leftrightarrow \text{[Ч], иногда [К]} & SH &\Leftrightarrow \text{[Ш]} && AL \Leftrightarrow \text{[O]}  \\
			OO &\Leftrightarrow \text{[У]} & EE &\Leftrightarrow \text{[И]} && OU \& OW \Leftrightarrow \text{[АУ], [У]} \\
			*ING &\Leftrightarrow \text{[ИН]} &
			KH &\Leftrightarrow \text{[Х]} &&
			*EW* \Leftrightarrow \text{[*ЬЮ*]} \\
			*E &\Leftrightarrow \text{не читается} &
			TH &\Leftrightarrow \text{среднее между [Д] и [С]} &&
			*GH(T)* \Leftrightarrow \text{*(Т)*} \\ 
			TION & \Leftrightarrow \text{[ШН]} && SION \Leftrightarrow \text{[ШН]} && CK \Leftrightarrow \text{[К]} \\
			KN* & \Leftrightarrow \text{[Н*]} && QU \Leftrightarrow \text{[КВ]} && PH \Leftrightarrow \text{[Ф]}
		\end{align*}
	\end{theorem}

	\begin{example}[Examples of pronouncement]
		\begin{align*}
			\text{Chair [чейр] (стул)} && \text{Cash [кэш] (наличные деньги)} \\
			\text{Book [бук] (книга)} && \text{To sleep [ту слип] (спать)} \\
			\text{Group [груп] (группа)}  && \text{Enough [инаф] (достаточно)} \\
			\text{Tower [тауэр] (башня)} && \text{New [нью] (новый)} \\
			\text{Call [кол] (телефонный звонок)} && \text{drawing [дроуин] (рисование)} \\
			\text{This [th-ис] (это)} && \text{Thing [th-инг] (вещь)} \\
			\text{Light [лайт] (свет)} && \text{Table [тейбл] (стол)} \\
			\text{Revolution [революшн] (революция)} && \text{Profession [профешн] (профессия)} \\
			\text{Duck [дак] (утка)} && \text{Knowledge [ноуледж] (знания)} \\
			\text{Quality [кволити] (качество)} && \text{Photography [фотографи] (фотография)}
		\end{align*}
	\end{example}
	
	\section{Personal Pronouns, verbs and basic structure of sentences}
	
	\begin{theorem}[Personal pronouns]
		\label{th: personal_pronouns}
		\index{personalpronouns}
		Личные местоимения:
		\begin{align*}
			\text{Single} && \text{Plural}  \\
			\hline
			\text{I} \Leftrightarrow \text{Я} && \text{We} \Leftrightarrow \text{Мы} \\
			\text{You} \Leftrightarrow \text{Ты} && \text{You} \Leftrightarrow \text{Вы} \\
			\text{He} \Leftrightarrow \text{Он} \\
			\text{She} \Leftrightarrow \text{Она} && \text{They} \Leftrightarrow \text{Они} \\
			\text{It} \Leftrightarrow \text{Это} \\
		\end{align*}
	\end{theorem}

	\begin{remark}
		В английском языке нет понятия "Род". У всех объектов общий род. Местоимения He и She лишь указывают на половую принадлежность человека.
	\end{remark}

	\begin{theorem}
		\label{th: to_be}
		\index{tobe}
		To be - быть:
		\begin{align*}
			\text{Single} && \text{Plural}  \\
			\hline
			\text{I am} && \text{We are} \\
			\text{You are} && \text{You are}\\
			\text{He is}\\
			\text{She is} && \text{They are} \\
			\text{It is} \\
		\end{align*}
	\end{theorem}

	\begin{theorem} Basic structure of sentences
	\label{th: basic_structure}
	\index{basicstructure}
	
		To do (did) - делать
	
		To work - работать
		
		To see (saw) - видеть

			\begin{tabular}{|c|c|c|c|c|}
				\hline
				& Утверждение & Отрицание & Вопрос & Отр. вопрос \\ \hline
				Настоящее & \vtop{\hbox{\strut I work}\hbox{\strut He works}\hbox{\strut I see}} &\vtop{\hbox{\strut I do not work}\hbox{\strut He does not work}} & \vtop{\hbox{\strut Do I work?}\hbox{\strut Does he work?}} & \vtop{\hbox{\strut Do not I work?}\hbox{\strut Does not he work?}} \\ \hline
				Прошедшее & \vtop{\hbox{\strut I worked}\hbox{\strut I saw}} & \vtop{\hbox{\strut I did not work}\hbox{\strut I did not see}} & \vtop{\hbox{\strut Did I work?}\hbox{\strut Did he see?}} & \vtop{\hbox{\strut Did not I work?}\hbox{\strut Did not he see?}}\\ \hline
				Будущее & \vtop{\hbox{\strut I will work}\hbox{\strut He will work}} & \vtop{\hbox{\strut I will not work}\hbox{\strut He will not work}} & \vtop{\hbox{\strut Will I work?}\hbox{\strut Will he work?}} & \vtop{\hbox{\strut Will not I work?}\hbox{\strut Will not he work?}}\\ \hline
			\end{tabular}
	
	\end{theorem}

	\newpage
	
	\begin{example}Some verbs
		\label{def:some_verbs}
		\index{some verbs}
		\begin{itemize}
			\item Go (went) -- идти, передвигаться
			\item Come (came) - идти, двигаться
			\item Become (became) -- приходить, становиться чем-либо
			\item Do (did) -- делать
			\item Speak (spoke) -- говорить
			\item Talk -- говорить, вести диалог	
			\item Live -- жить
			\item Want -- хотеть
			\item Give (gave) -- давать
			\item Take (took) -- брать
			\item Put (put) -- поместить
			\item Work -- работать
			\item Love -- любить
			\item Run (ran) -- бежать
			\item Walk -- идти пешком
			\item See (saw) -- видеть
			\item Look -- смотреть
			\item Listen to -- слушать
			\item Hear (heard) -- слышать
			\item Meet (met) -- встречать
		\end{itemize}
	\end{example}

\begin{theorem}
	\label{th: pronouns2}
	\index{pronouns2}
	Subjective and objective pronouns:
	\begin{align*}
		\text{Subject} && \text{Object}  \\
		\hline
		\text{I} && \text{Me} \\
		\text{You} && \text{You}\\
		\text{He} && \text{Him}\\
		\text{She} && \text{Her} \\
		\text{We} && \text{Us} \\
		\text{They} && \text{Them}
	\end{align*}
\end{theorem}

	\newpage
	
	\section{Text}
	
	\textbf{Jeff Wilson, a new employee, meets Mark and Sandra}
	\begin{definition}
		\begin{tabular}{ll}
			Jeff: & \text{Good morning.} \\
			& \text{I am Jeff Wilson} \\
			Mark: & \text{Good morning,} \\
			& \text{my name is Mark de Kruiff.} \\
			& \text{Are you the new employee?} \\
			Jeff: & \text{Yes, I am the new employee.} \\
			& \text{I am American,} \\
			& \text{I am from New York.} \\
			& \text{I live now in The Netherlands} \\
			Mark: & \text{Where do you live?} \\
			Jeff: & \text{I live in Utrecht} \\
			& \text{And you?} \\
			Mark: & \text{I am Dutchman} \\
			& \text{and I live in Eindhoven.} \\
			& \text{Do you speak Dutch?} \\
			Jeff: & Yes, I speak Dutch and English. \\
			Mark: & Let me introduce you: \\
			& This is Sandra van Wittem. \\
			& She works also in Euroline \\
			& as a secretary \\
			Jeff: & Nice to meet you \\
			Mark: & Sandra, this is a new employee. \\
			& His name is Jeff Wilson \\
			Sandra: & Nice to meet you. \\
			& Welcome to Euroline. \\
			Jeff: & Thank you. \\
			& Are you not Dutch? \\
			Sandra: & No, I am from Belgium, \\
			& I am Belgian, \\
			& and now I live in The Netherlands \\
			Mark: & Do you want a coffee? \\
			Jeff and Sandra: & Yes, sure. \\
			Mark: & You are welcome\\
		\end{tabular}
	\end{definition}

\begin{example} Vocabulary to the text
	\label{def:vocabulary_1}
	\index{vocabulary_1}
	\begin{itemize}
		\item Good morning -- Доброе утро
		\item Name - Имя
		\item New -- новый
		\item Employee -- работник
		\item American -- американец
		\item To live -- жить	
		\item From -- из
		\item In -- в
		\item Where -- где
		\item And -- и
		\item Dutchman -- нидерландец
		\item To speak -- говорить
		\item Let me -- позволь мне
		\item To introduce -- представлять, знакомить
		\item Belgian -- бельгиец
		\item To work -- работать
		\item Now -- сейчас
		\item Also -- также
		\item As -- как (работает как секретарь)
		\item Secretary -- секретарь
		\item Nice to meet you -- рад познакомиться, рад встретить тебя
		\item To want -- хотеть
		\item Coffee -- кофе
		\item Yes, sure -- да, конечно
		\item You are welcome -- пожалуйста, ответ на спасибо
	\end{itemize}
\end{example}

	Questions to the text:

	\begin{enumerate}
		\item Is Jeff Wilson American?
		\item  Is Mark de Kruiff American?
		\item Does Jeff speak Dutch?
		\item Is Sandra from The Netherlands?
		\item Where does Sandra work?
		\item Does Jeff live in Amsterdam?
		\item Is Sandra a secretary?
	    \item Who is a new employee?
	    \item Does Sandra drink a coffee?		
	\end{enumerate}
	
	%---------------------------------------------------------------------------
	% Bibliography
	%---------------------------------------------------------------------------
	
	\addcontentsline{toc}{chapter}{\textcolor{tssteelblue}{Literature}}
	
	%---------------------------------------------------------------------------
	% Index
	%---------------------------------------------------------------------------
	
	\printindex
	
\end{document}
