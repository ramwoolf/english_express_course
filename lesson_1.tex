\documentclass{tstextbook}

\begin{document}
	
	%---------------------------------------------------------------------------
	% Chapters
	%---------------------------------------------------------------------------
	
	%---------------------------------------------------------------------------
	\chapter{Lesson 1}
	
	\section{Alphabet and transcription}
	
	\textcolor{orange}{\LARGE A B C D E F G H I J K L M N O P Q R S T U V W X Y Z}
	
	\begin{theorem}[Transcription]
		\label{th:transcription}
		\index{trancription}
		Основные правила транскрипции:
		\begin{align*}
			CH &\Leftrightarrow \text{[Ч], иногда [К]} & SH &\Leftrightarrow \text{[Ш]} && AL \Leftrightarrow \text{[O]}  \\
			OO &\Leftrightarrow \text{[У]} & EE &\Leftrightarrow \text{[И]} && OU \& OW \Leftrightarrow \text{[АУ], [У]} \\
			*ING &\Leftrightarrow \text{[ИН]} &
			KH &\Leftrightarrow \text{[Х]} &&
			*EW* \Leftrightarrow \text{[*ЬЮ*]} \\
			*E &\Leftrightarrow \text{не читается} &
			TH &\Leftrightarrow \text{среднее между [Д] и [С]} &&
			*GH(T)* \Leftrightarrow \text{*(Т)*} \\ 
			TION & \Leftrightarrow \text{[ШН]} && SION \Leftrightarrow \text{[ШН]} && CK \Leftrightarrow \text{[К]} \\
			KN* & \Leftrightarrow \text{[Н*]} && QU \Leftrightarrow \text{[КВ]} && PH \Leftrightarrow \text{[Ф]}
		\end{align*}
	\end{theorem}

	\begin{example}[Examples of pronouncement]
		\begin{align*}
			\text{Chair [чейр] (стул)} && \text{Cash [кэш] (наличные деньги)} \\
			\text{Book [бук] (книга)} && \text{To sleep [ту слип] (спать)} \\
			\text{Group [груп] (группа)}  && \text{Enough [инаф] (достаточно)} \\
			\text{Tower [тауэр] (башня)} && \text{New [нью] (новый)} \\
			\text{Call [кол] (телефонный звонок)} && \text{drawing [дроуин] (рисование)} \\
			\text{This [th-ис] (это)} && \text{Thing [th-инг] (вещь)} \\
			\text{Light [лайт] (свет)} && \text{Table [тейбл] (стол)} \\
			\text{Revolution [революшн] (революция)} && \text{Profession [профешн] (профессия)} \\
			\text{Duck [дак] (утка)} && \text{Knowledge [ноуледж] (знания)} \\
			\text{Quality [кволити] (качество)} && \text{Photography [фотографи] (фотография)}
		\end{align*}
	\end{example}
	
	\section{Personal Pronouns, verbs and basic structure of sentences}
	
	\begin{theorem}[Personal pronouns]
		\label{th: personal_pronouns}
		\index{personalpronouns}
		Личные местоимения:
		\begin{align*}
			\text{Single} && \text{Plural}  \\
			\text{I} \Leftrightarrow \text{Я} && \text{We} \Leftrightarrow \text{Мы} \\
			\text{You} \Leftrightarrow \text{Ты} && \text{You} \Leftrightarrow \text{Вы} \\
			\text{He} \Leftrightarrow \text{Он} \\
			\text{She} \Leftrightarrow \text{Она} && \text{They} \Leftrightarrow \text{Они} \\
			\text{It} \Leftrightarrow \text{Это} \\
		\end{align*}
	\end{theorem}

	\begin{remark}
		В английском языке нет понятия "Род". У всех объектов общий род. Местоимения He и She лишь указывают на половую принадлежность человека.
	\end{remark}

	\begin{theorem}
		\label{th: to_be}
		\index{tobe}
		To be - быть:
		\begin{align*}
			\text{Single} && \text{Plural}  \\
			\text{I am} && \text{We are} \\
			\text{You are} && \text{You are}\\
			\text{He is}\\
			\text{She is} && \text{They are} \\
			\text{It is} \\
		\end{align*}
	\end{theorem}

	\begin{theorem} Basic structure of sentences
	\label{th: basic_structure}
	\index{basicstructure}
	
		Глагол "делать" - to do (did)
	
		Глагол "работать" - to work
		
		Глагол "видеть" - to see (saw)
	
%		\begin{table}
			\begin{tabular}{|c|c|c|c|c|}
				\hline
				& Утверждение & Отрицание & Вопрос & Отр. вопрос \\ \hline
				Настоящее & \vtop{\hbox{\strut I work}\hbox{\strut He works}\hbox{\strut I see}} &\vtop{\hbox{\strut I do not work}\hbox{\strut He does not work}} & \vtop{\hbox{\strut Do I work?}\hbox{\strut Does he work?}} & \vtop{\hbox{\strut Do not I work?}\hbox{\strut Does not he work?}} \\ \hline
				Прошедшее & \vtop{\hbox{\strut I worked}\hbox{\strut I saw}} & \vtop{\hbox{\strut I did not work}\hbox{\strut I did not see}} & \vtop{\hbox{\strut Did I work?}\hbox{\strut Did he see?}} & \vtop{\hbox{\strut Did not I work?}\hbox{\strut Did not he see?}}\\ \hline
				Будущее & \vtop{\hbox{\strut I will work}\hbox{\strut He will work}} & \vtop{\hbox{\strut I will not work}\hbox{\strut He will not work}} & \vtop{\hbox{\strut Will I work?}\hbox{\strut Will he work?}} & \vtop{\hbox{\strut Will not I work?}\hbox{\strut Will not he work?}}\\ \hline
			\end{tabular}
%		\end{table}
	
	\end{theorem}

	
	\section{Third section}
	
	Now let's move on to the definition of the real number system. This
	may be defined in a multitude of ways, one of which is to think about
	a real number as a rational Cauchy sequence, or rather the equivalence
	class of Cauchy sequences ``converging to'' that number.
	
	\begin{definition}[The real numbers $\mathbb{R}$]
		\label{def:realnumbers}
		\index{real numbers}
		The real numbers $\mathbb{R}$ is the set of all equivalence classes
		of rational Cauchy sequences.
	\end{definition}
	
	Now that this is settled, lets prove the completeness of the real
	number system.
	
	\begin{theorem}[The completeness of the real numbers]
		\label{th:realnumberscomplete}
		\index{completeness of the real numbers}
		Let $(x_n)_{n=0}^{\infty}$ be a sequence of real numbers.
		Then $(x_n)_{n=0}^{\infty}$ is convergent if and only if
		it is also a real Cauchy sequence.
	\end{theorem}
	\begin{proof}
		Write $x_m = [(x_{mn})_{n=0}^{\infty}]$ where
		$x_{mn}$ is the $n$th number in a rational Cauchy sequence
		representing the real number $x_m$. And so on\ldots.
	\end{proof}
	
	For further reading, there are several excellent works that one could
	cite, such as \cite{Tao2006,Turing1936}.
	
	\section*{Exercises}
	
	\begin{exercise}
		Let $A = \{1, 2, 3\}$ and $B = \{2, 3, 4\}$.
		Determine the following sets. \\
		(a) $A \cup B$ \quad
		(b) $A \cap B$ \quad
		(c) $A \setminus B$ \quad
		(d) $A \times B$
	\end{exercise}
	
	\begin{exercise}
		Let $A = \{1, 3, 5, 7, 9\}$ and $B = \{2, 4, 6, 8, 10\}$.
		Determine the following sets. \\
		(a) $A \cup B$ \quad
		(b) $A \cap B$ \quad
		(c) $A \setminus B$ \quad
		(d) $A \times B$
	\end{exercise}
	
	\begin{exercise}
		Let $A = \{1, 2, 3\}$, $B = \{2, 3, 4\}$ and $C = \{3, 4, 5\}$.
		Determine the following sets. \\
		(a) $A \cup B \cup C$ \quad
		(b) $A \cap B \cap C$ \quad
		(c) $(B \setminus A) \cap C$ \quad
		(d) $(A \times B) \times C$
	\end{exercise}
	
	\section*{Problem}
	
	\begin{problem}
		Interpret the following set definition (Russell's paradox) and discuss
		whether $X \in X$ or $X \notin X$:
		\begin{equation}
			X = \{x \mid x \notin x\}.
		\end{equation}
	\end{problem}
	
	\section*{Computer exercises}
	
	\begin{programming}
		Write a program that generates the sequence $(x_n)_{n=0}^{100}$
		for $x_n = n$.
	\end{programming}
	
	\begin{programming}
		Write a program that generates the odd numbers between $1$ and $100$.
	\end{programming}
	
	\begin{programming}
		Write a program that computes the sum $\sum_{n=0}^{100} x_n$
		for $x_n = n$.
	\end{programming}
	
	%---------------------------------------------------------------------------
	\chapter{Second chapter}
	
	\begin{summary}
		\blindtext
	\end{summary}
	
	\section{First section}
	\Blindtext
	
	\section{Second section}
	\Blindtext
	
	\section{Third section}
	\Blindtext
	
	%---------------------------------------------------------------------------
	\chapter{Third chapter}
	
	\begin{summary}
		\blindtext
	\end{summary}
	
	\section{First section}
	\Blindtext
	
	\section{Second section}
	\Blindtext
	
	\section{Third section}
	\Blindtext
	
	%---------------------------------------------------------------------------
	% Bibliography
	%---------------------------------------------------------------------------
	
	\addcontentsline{toc}{chapter}{\textcolor{tssteelblue}{Literature}}
	
	%---------------------------------------------------------------------------
	% Index
	%---------------------------------------------------------------------------
	
	\printindex
	
\end{document}
