\documentclass{tstextbook}

\begin{document}
	
	%---------------------------------------------------------------------------
	% Chapters
	%---------------------------------------------------------------------------
	
	%---------------------------------------------------------------------------
	\chapter{Lesson 2}
	
	\section{Basic structure of sentences}

	\begin{theorem} Basic structure of sentences
	\label{th: basic_structure}
	\index{basicstructure}
	
		To do (did) - делать
	
		To work - работать
		
		To see (saw) - видеть

			\begin{tabular}{|c|c|c|c|c|}
				\hline
				& Утверждение & Отрицание & Вопрос & Отр. вопрос \\ \hline
				Настоящее & \vtop{\hbox{\strut I work}\hbox{\strut He works}\hbox{\strut I see}} &\vtop{\hbox{\strut I do not work}\hbox{\strut He does not work}} & \vtop{\hbox{\strut Do I work?}\hbox{\strut Does he work?}} & \vtop{\hbox{\strut Do not I work?}\hbox{\strut Does not he work?}} \\ \hline
				Прошедшее & \vtop{\hbox{\strut I worked}\hbox{\strut I saw}} & \vtop{\hbox{\strut I did not work}\hbox{\strut I did not see}} & \vtop{\hbox{\strut Did I work?}\hbox{\strut Did he see?}} & \vtop{\hbox{\strut Did not I work?}\hbox{\strut Did not he see?}}\\ \hline
				Будущее & \vtop{\hbox{\strut I will work}\hbox{\strut He will work}} & \vtop{\hbox{\strut I will not work}\hbox{\strut He will not work}} & \vtop{\hbox{\strut Will I work?}\hbox{\strut Will he work?}} & \vtop{\hbox{\strut Will not I work?}\hbox{\strut Will not he work?}}\\ \hline
			\end{tabular}
	
	\end{theorem}

\begin{theorem}
	\label{th: pronouns2}
	\index{pronouns2}
	Subjective, objective and possessive pronouns:
	\begin{align*}
		\text{Subject} && \text{Object} && \text{Possessive} \\
		\hline
		\text{I} && \text{Me} && \text{My} \\
		\text{You} && \text{You} && \text{Your(s)} \\
		\text{He} && \text{Him} && \text{His} \\
		\text{She} && \text{Her}  && \text{Her} \\
		\text{We} && \text{Us}  && \text{Our(s)} \\
		\text{They} && \text{Them} && \text{Their} 
	\end{align*}
\end{theorem}

\begin{theorem}
	\label{th: some_prepositions}
	\index{some_prepositions}
	Some prepositions:
	\begin{align*}
		& On \\
		& \Downarrow \\
		To \Rightarrow & \text{In \faCreativeCommonsBy} \Rightarrow From \\
		\text{With \faPeopleCarry}  && \text{Without \faMale \space \faMale }
	\end{align*}

\end{theorem}

\begin{theorem}
	\label{th: questions}
	\index{questions}
	questions:
	\begin{itemize}
		\item What? - Что?
		\item Where? - Где? Куда?
		\item When? - Когда?
		\item Why? - Почему?
		\item Who? - Кто?
		\item Which? - Какой? Который?
	\end{itemize}

	What do you see?
	
	Why did he work?
\end{theorem}
	
	\begin{example}Some verbs
		\label{def:some_verbs}
		\index{some verbs}
		\begin{itemize}
			\item Go (went) -- идти, передвигаться
			\item Come (came) - идти, двигаться
			\item Become (became) -- приходить, становиться чем-либо
			\item Do (did) -- делать
			\item Speak (spoke) -- говорить
			\item Talk -- говорить, вести диалог	
			\item Live -- жить
			\item Want -- хотеть
			\item Give (gave) -- давать
			\item Take (took) -- брать
			\item Put (put) -- поместить
			\item Work -- работать
			\item Love -- любить
			\item Run (ran) -- бежать
			\item Walk -- идти пешком
			\item See (saw) -- видеть
			\item Look -- смотреть
			\item Listen to -- слушать
			\item Hear (heard) -- слышать
			\item Meet (met) -- встречать
			\item Have (had) -- иметь
			\item Know (knew) -- знать
			\item Get (got) -- получать
			\item Like -- находить приятным
			\item Eat (ate) -- Есть еду
			\item Drink (drank) -- Пить
		\end{itemize}
	\end{example}

	\begin{programming}
		Build the phrases:
		\begin{enumerate}
			\item Я говорю им
			\item Они не слышали меня
			\item Почему ты слушаешь его?
			\item Мы не видим её
			\item Они получали это от нас
			\item Куда ты хочешь поехать?
			\item Она будет знать это
			\item Я не ел
			\item Вам это нравится?
			\item Он не придёт к нам
			\item Я и ты разговаривали с ними
			\item Кто ты и откуда ты?
			\item Я слушал их
			\item Она положит это к нам
			\item Когда мы поедем к ним?
			\item Она приходит из офиса
			\item Вы брали их кофе
			\item Почему они бегут от меня?
			\item Ты узн\'aешь
			\item Я положил это на стол
			\item Куда ты это положил?
			\item Я гуляю и слушаю музыку
			\item Они жили и работали в городе
			\item Кто придёт к нам?
		\end{enumerate}
	\end{programming}

	\newpage
	
	\section{Text}
	
	\textbf{Jeff Wilson, a new employee, meets Mark and Sandra}
	\begin{definition}
		\begin{tabular}{ll}
			Jeff: & \text{Good morning.} \\
			& \text{I am Jeff Wilson} \\
			Mark: & \text{Good morning,} \\
			& \text{my name is Mark de Kruiff.} \\
			& \text{Are you the new employee?} \\
			Jeff: & \text{Yes, I am the new employee.} \\
			& \text{I am American,} \\
			& \text{I am from New York.} \\
			& \text{I live now in The Netherlands} \\
			Mark: & \text{Where do you live?} \\
			Jeff: & \text{I live in Utrecht} \\
			& \text{And you?} \\
			Mark: & \text{I am Dutchman} \\
			& \text{and I live in Eindhoven.} \\
			& \text{Do you speak Dutch?} \\
			Jeff: & Yes, I speak Dutch and English. \\
			Mark: & Let me introduce you: \\
			& This is Sandra van Wittem. \\
			& She works also in Euroline \\
			& as a secretary \\
			Jeff: & Nice to meet you \\
			Mark: & Sandra, this is a new employee. \\
			& His name is Jeff Wilson \\
			Sandra: & Nice to meet you. \\
			& Welcome to Euroline. \\
			Jeff: & Thank you. \\
			& Are you not Dutch? \\
			Sandra: & No, I am from Belgium, \\
			& I am Belgian, \\
			& and now I live in The Netherlands \\
			Mark: & Do you want a coffee? \\
			Jeff and Sandra: & Yes, sure. \\
			Mark: & You are welcome\\
		\end{tabular}
	\end{definition}

\begin{example} Vocabulary to the text
	\label{def:vocabulary_1}
	\index{vocabulary_1}
	\begin{itemize}
		\item Good morning -- Доброе утро
		\item Name - Имя
		\item New -- новый
		\item Employee -- работник
		\item American -- американец
		\item To live -- жить	
		\item From -- из
		\item In -- в
		\item Where -- где
		\item And -- и
		\item Dutchman -- нидерландец
		\item To speak -- говорить
		\item Let me -- позволь мне
		\item To introduce -- представлять, знакомить
		\item Belgian -- бельгиец
		\item To work -- работать
		\item Now -- сейчас
		\item Also -- также
		\item As -- как (работает как секретарь)
		\item Secretary -- секретарь
		\item Nice to meet you -- рад познакомиться, рад встретить тебя
		\item To want -- хотеть
		\item Coffee -- кофе
		\item Yes, sure -- да, конечно
		\item You are welcome -- пожалуйста, ответ на спасибо
	\end{itemize}
\end{example}

	
	%---------------------------------------------------------------------------
	% Bibliography
	%---------------------------------------------------------------------------
	
	\addcontentsline{toc}{chapter}{\textcolor{tssteelblue}{Literature}}
	
	%---------------------------------------------------------------------------
	% Index
	%---------------------------------------------------------------------------
	
	\printindex
	
\end{document}
